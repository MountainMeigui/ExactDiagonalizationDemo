\documentclass[12pt,a4paper]{article}

% Packages
\usepackage[utf8]{inputenc}
\usepackage[T1]{fontenc}
\usepackage{amsmath,amssymb,amsthm}
\usepackage{braket}
\usepackage{graphicx}
\usepackage{hyperref}
\usepackage{natbib}
\usepackage{booktabs}
\usepackage{geometry}
\geometry{margin=1in}
\usepackage{setspace}
\onehalfspacing

% Theorem environments
\newtheorem{theorem}{Theorem}
\newtheorem{lemma}{Lemma}
\newtheorem{proposition}{Proposition}
\newtheorem{corollary}{Corollary}
\theoremstyle{definition}
\newtheorem{definition}{Definition}
\newtheorem{example}{Example}
\theoremstyle{remark}
\newtheorem{remark}{Remark}

% Title and authors
\title{Counter propagating FQH edge modes on an annulus}
\author{Vered Cohen}
\date{\today}

\begin{document}

\maketitle

\begin{abstract}
    We study a system of interacting spin-polarized electrons confined to an annulus in a
strong magnetic field.
\end{abstract}

\section{Introduction - Landau Levels in the Symmetric Gauge}
\label{sec:introduction}

The quantum Hall effects describe the behavior of 2d electrons in strong magnetic fields. There
are a few necessary components for the understanding of these effects in full, but a good place
to start is the Hamiltonian describing an electron in a constant magnetic field:

\[
H = \frac{(\mathbf{p} - e\mathbf{A})^2}{2m_e}
\]

This Hamiltonian is exactly solvable, but in order to write explicit eigenfunctions, we must
choose a gauge. For reasons that will become clear later on, we choose the symmetric gauge:

\[
\mathbf{A} = \left( -\frac{By}{2}, \frac{Bx}{2}, 0 \right).
\]
In this gauge, the angular momentum Lz is conserved:
\[
L_z = i\hbar \left( x \frac{\partial}{\partial y} - y \frac{\partial}{\partial x} \right)
\]
\[
[H, L_z] = 0,
\]
thus the eigenstates of the Hamiltonian can be labeled by the angular momentum $\hbar m$ ( for any
positive integer $m$). The spectrum of this Hamiltonian is:

\[
E_{n,m} = \hbar \omega_c \left( n + \frac{1}{2} \right), \quad n \in \{0, 1, 2, ...\},
\]
where $\omega_c = \frac{eB}{m_e}$ is the cyclotron frequency.
Note the the spectrum doesn’t depend on m, so every energy level is infinitely degenerate.
These energy levels are called Landau levels, and although an explicit form can be written for any
general level using the appropriate Hermit polynomial, we will focus on the lowest Landau level
(LLL), i.e. $n = 0$.
Using complex coordinates to describe the plane, $z = x - iy$, the eigenfunctions of the LLL
can be written as:

\[  
\psi_m(z) = \frac{1}{\sqrt{2\pi 2^m m! l_B^{2}}} z^m e^{-\frac{|z|^2}{4l_B^2}} ,
\]

where we define the magnetic length $l_B = \sqrt{\frac{\hbar}{eB}}$ for convenience.

A general state in the LLL is any (normalized) linear combination of $\{\psi _m\}_{m=0}^\infty$. 
In other words, the LLL is the space of all functions:
\[
\psi (z) = f(z) e^{-\frac{|z|^2}{4l_B^2}}, 
\]

where $f(z)$ is a holomorphic function.

\subsection{spatial structure of the LLL eigenfunction}
Each eigenfunction $\psi_m$ is peaked at a radius $r = \sqrt{2m} l_B$, and decays
exponentially for larger or smaller radii. Thus, we can think of each eigenfunction as being localized
on a ring of radius $\sqrt{2m} l_B$.

\section{Model}
\label{sec:model}

\subsection{Choosing the Hilbert space}
We study 2d electrons in a strong magnetic field, so as a first approximation we restrict
our Hilbert space to the lowest Landau level (LLL). Since our system is a physical annulus, we take advantage of
the spatial structure of the LLL in the symmetric gauge (described in Sec. 2.2.1) and truncate
our Hilbert space to include only LLL eigenfunctions localized on our annulus. More precisely,
choosing two radii $r_{\min} < r_{\max}$, such that $r_{\min} = \sqrt{2M_{\min}}l_B$ and $r_{\max} = \sqrt{2M_{\max}}l_B$ for some
integers $M_{\min}, M_{\max}$, the Hilbert space we choose is all the LLL eigenstates with angular momenta
between $M_{\min}$ and $M_{\max}$.

\subsection{The interacting Hamiltonian to model}

The full Hamiltonian of our system is:
\[
H = H_B + H_\mathrm{interactions} + H_\mathrm{confining\ potential}
\]
where $H_B$ is the kinetic energy term in the presence of the magnetic field (which is constant
in the LLL and thus we ignore it from this point on), $H_\mathrm{interactions}$ is the two-body interaction that is the parent Hamiltonian to the Laughlin state, and $H_\mathrm{confining\ potential}$ is a one-body confining potential that keeps the electrons
localized on the annulus.

We Take the interaction term to be

% \[
% H_\mathrm{int} = \sum_{i<j} \nabla ^2 \delta(r_i-r_j), 
% \]


\[
H_\mathrm{int} = \int dz \Psi ^\dagger (z_1)\Psi ^\dagger (z_2) \nabla ^2 \delta(r_1-r_2) \Psi (z_2)\Psi (z_1), 
\]

And the confining potential to be
\begin{align*}
H_\mathrm{confining\ potential}&=\frac{\beta}{l_{B}^{2}}\int dz\left|r^{2}-r_{0}^{2}\right|\Psi^{\dagger}\left(z\right)\Psi\left(z\right)\\
m_{0}&=\frac{M_{min}+M_{max}}{2},\\ 
r_{m_{0}}&=\sqrt{2\left(m_{0}+1\right)}l_{B}\\
z&=re^{-i\theta},    
\end{align*}

\subsection{The matrix elements in our Hilbert space}
To construct the matrix elements of the Hamiltonian in our chosen Hilbert space, we calculate the
matrix elements of each term separately.

The confining potential term matrix elements are given by:

\begin{align*}
    \braket{m|H_\mathrm{confining\ potential}|n}&= \\
    &\delta_{m-n}l_{B}^{2}\left[\left(2\left(m+1\right)-x_{0}^{2}\right)\frac{\left[\Gamma\left(m+1,\frac{x_{0}^{2}}{2}\right)-\gamma\left(m+1,\frac{x_{0}^{2}}{2}\right)\right]}{\Gamma\left(m+1\right)}+4\frac{x_{0}^{2m+2}}{m!2^{m+1}}e^{-\frac{x_{0}^{2}}{2}}\right]
\end{align*}


As for the interaction term: the parent Hamiltonian can be written using Haledane's pseudopotentials formalism, for the pseudopotentials: 
\[
V_{m}=\begin{cases}
1 & m<\nu^{-1}\\
0 & m\ge\nu^{-1}
\end{cases}
\]

Where, since we are considering fermions, we can take $V_{m}$ for even $m$ to zero. Also, Since we will focus on $\nu =\frac{1}{3}$, we only need to consider $V_{1}\ne 0$.

Since this is a two-particle interaction, we need matrix elements between two-particle states to describe the full interactions matrix in the many-body Hilbert space.

We will need to change basis between particle angular momenta $\ket{m_{1},m_{2}}$, and relative and center of mass (COM) angular momenta $\ket{M,m}$, since that is the basis in which the pseudopotentials are given.

The final result is: 
\begin{align*}
&\braket{m_{1},m_{2}|H_{int}|m_{3},m_{4}}= \\
&\delta_{m_{1}+m_{2}-m_{3}-m_{4}}\sum_{m=0}^{m_{1}+m_{2}}V_{m}\braket{m_{1},m_{2}|m_{1}+m_{2}-m,m}\braket{m_{1}+m_{2}-m,m|m_{3},m_{4}}    
\end{align*}

Where $\ket{m_{1}+m_{2}-m,m}$ is a state with well defined COM angular momentum $M=m_{1}+m_{2}-m$ and relative angular momentum $m$.


The overlap amplitudes $\braket{m_{1},m_{2}|M,m}$ are given by:
\begin{align*}
\braket{m_{1},m_{2}|M,m}&=\delta_{m_{1}+m_{2}-M-m}\frac{\alpha\left(m_{1},m_{2},m\right)}{2\sqrt{2^{m_{1}+m_{2}+2}m_{1}!m_{2}!}} \\
\alpha\left(m_{1},m_{2},m\right)&=\sqrt{m!\left(m_{1}+m_{2}-m\right)!}\sum_{l=0}^{m}\left(-1\right)^{l} \left( \binom{m_1}{m-l} \binom{m_2}{l} - \binom{m_2}{m_-l} \binom{m_1}{l} \right)
\end{align*}

\subsection{Using these elements in a many body hamiltonian}
To construct the many body Hamiltonian matrix in our chosen Hilbert space, we use the
matrix elements calculated above, and second quantization formalism.
The full Hamiltonian in second quantization is given by:
\[
H = \sum_{m,n} \braket{m|H_\mathrm{confining\ potential}|n} c_m^\dagger c_n + \frac{1}{2} \sum_{m_1, m_2, m_3, m_4} \braket{m_1, m_2|H_\mathrm{int}|m_3, m_4} c_{m_1}^\dagger c_{m_2}^\dagger c_{m_4} c_{m_3}
\]

Where $c_m^\dagger$ and $c_m$ are the creation and annihilation operators for an electron in the
LLL eigenstate with angular momentum $\hbar m$.

We must remember to account for fermionic anti-commutation relations when constructing the many-body Hamiltonian matrix.

\subsection{Minimal size of Hilbert space}
To properly capture the physics of $N$ electrons in the Laughlin state at filling factor $\nu = \frac{1}{3}$ on an annulus, we need to choose our Hilbert space size appropriately.
This dimension turn out to be at least:
\[ 
\mathrm{dim}H = 3\cdot (N-1) + 1
\]

\section{Results}
\label{sec:results}

Present your findings.

\section{Conclusion}
\label{sec:conclusion}

Use Claude. He is the best.

\bibliographystyle{plain}
\bibliography{references}

\end{document}
