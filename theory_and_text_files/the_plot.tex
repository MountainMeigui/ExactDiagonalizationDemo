\documentclass[14pt,a4paper]{extarticle}

% Packages
\usepackage[utf8]{inputenc}
\usepackage[T1]{fontenc}
\usepackage{amsmath,amssymb,amsthm}
\usepackage{braket}
\usepackage{graphicx}
\usepackage{hyperref}
\usepackage{natbib}
\usepackage{booktabs}
\usepackage{geometry}
\geometry{margin=1in}
\usepackage{setspace}
\onehalfspacing

% Theorem environments
\newtheorem{theorem}{Theorem}
\newtheorem{lemma}{Lemma}
\newtheorem{proposition}{Proposition}
\newtheorem{corollary}{Corollary}
\theoremstyle{definition}
\newtheorem{definition}{Definition}
\newtheorem{example}{Example}
\theoremstyle{remark}
\newtheorem{remark}{Remark}

% Title and authors
\title{A student's struggle with many-body electron interactions}
\author{Vered Cohen}
% \date{\today}

\begin{document}

\maketitle

\begin{abstract}
    We study a system of interacting spin-polarized electrons confined to an annulus in a
strong magnetic field. We need the help of our computer to diagonalize the Hamiltonian.
We ask Claude to help us write the code.
Then we ask him to help us edit our paper and create pretty figures.
\end{abstract}

\section{The many-body Hamiltonian in question}
\label{sec:many_body_hamiltonian}

A lot of sweat and tears must first go into translating the physical system into a matrix that our computer can understand.

The full Hamiltonian of our system is:
\begin{align*}
H &= \sum_{m,n} \braket{m|H_\mathrm{confining\ potential}|n} c_m^\dagger c_n \\
&\quad + \frac{1}{2} \sum_{m_1, m_2, m_3, m_4} \braket{m_1, m_2|H_\mathrm{int}|m_3, m_4} c_{m_1}^\dagger c_{m_2}^\dagger c_{m_4} c_{m_3}
\end{align*}

Where $c_m^\dagger$ and $c_m$ are the creation and annihilation operators for an electron in the
LLL eigenstate with angular momentum $\hbar m$ (or as far as the computer is concerned, just a set of operators with anti-commutation relations).

\vspace{1em}
\underline{The confining potential term matrix elements are given by:}

\begin{align*}
    \braket{m|H_\mathrm{confining\ potential}|n} &= \delta_{m-n}V\Bigg[ \\
    &\left(2\left(m+1\right)-x_{0}^{2}\right) \frac{\left[\Gamma\left(m+1,\frac{x_{0}^{2}}{2}\right)-\gamma\left(m+1,\frac{x_{0}^{2}}{2}\right)\right]}{\Gamma\left(m+1\right)} \\
    &\quad +4\frac{x_{0}^{2m+2}}{m!2^{m+1}}e^{-\frac{x_{0}^{2}}{2}}\Bigg]
\end{align*}

where $\Gamma(a,x)$ and $\gamma(a,x)$ are the upper and lower incomplete gamma functions respectively:
\begin{align*}
\Gamma(a,x) &= \int_x^\infty t^{a-1} e^{-t} dt \\
\gamma(a,x) &= \int_0^x t^{a-1} e^{-t} dt, 
\end{align*}

$\Gamma(a)$ is the standard gamma function:
\[
\Gamma(a) = \int_0^\infty t^{a-1} e^{-t} dt,
\]

and $V, x_0$ are the confining potential strength and the normalized radius of the annulus describing the geometry of our system.

\vspace{1em}
\underline{The interaction term matrix elements are given by:}
\begin{align*}
\braket{m_{1},m_{2}|H_{int}|m_{3},m_{4}} &= \delta_{m_{1}+m_{2}-m_{3}-m_{4}}V_{1} \\
&\quad \times \braket{m_{1},m_{2}|m_{1}+m_{2}-1,1} \\
&\quad \times \braket{m_{1}+m_{2}-1,1|m_{3},m_{4}}    
\end{align*}
% \begin{align*}
% &\braket{m_{1},m_{2}|H_{int}|m_{3},m_{4}}= \\
% &\delta_{m_{1}+m_{2}-m_{3}-m_{4}}\sum_{m=0}^{m_{1}+m_{2}}V_{m}\braket{m_{1},m_{2}|m_{1}+m_{2}-m,m}\braket{m_{1}+m_{2}-m,m|m_{3},m_{4}}    
% \end{align*}
Where $\ket{m_{1}+m_{2}-m,m}$ is a state with well defined COM angular momentum $M=m_{1}+m_{2}-m$ and relative angular momentum $m$, and $V_1$ is a parameter describing the strength of the interaction.

Or, as far as the computer needs to know, the overlap amplitudes $\braket{m_{1},m_{2}|M,m}$ are just given by:
\begin{align*}
\braket{m_{1},m_{2}|M,m}&=\delta_{m_{1}+m_{2}-M-m}\frac{\alpha\left(m_{1},m_{2},m\right)}{2\sqrt{2^{m_{1}+m_{2}+2}m_{1}!m_{2}!}} \\
\alpha\left(m_{1},m_{2},m\right)&=\sqrt{m!\left(m_{1}+m_{2}-m\right)!} \\
&\quad \times \sum_{l=0}^{m}\left(-1\right)^{l} \left( \binom{m_1}{m-l} \binom{m_2}{l} - \binom{m_2}{m-l} \binom{m_1}{l} \right).
\end{align*}

\section{The many-body basis we choose}
\label{many_body_basis}
To represent the many-body Hamiltonian as a matrix, we need to choose a basis for the many-body Hilbert space.

The single electron states in our Hilbert space are the LLL eigenstates with angular momenta $\hbar m$, represented with creation operators $c_m^\dagger$, where $m$ ranges from $m_\mathrm{min}$ to $m_\mathrm{max}$ (this is a good description for our case since we want to study electrons confined to an annulus, and the angular momentum resolved states are localized along rings of radii $r_m = \sqrt{2m}l_B$).
The many-body basis states are constructed as Slater determinants of these single electron states, which can be represented in second quantization as:
\begin{align*}
\ket{n_{m_\mathrm{min}}, n_{m_\mathrm{min}+1}, \ldots, n_{m_\mathrm{max}}} &= (c_{m_\mathrm{max}}^\dagger)^{n_{m_\mathrm{max}}} \ldots \\
&\quad \times (c_{m_\mathrm{min}+1}^\dagger)^{n_{m_\mathrm{min}+1}} (c_{m_\mathrm{min}}^\dagger)^{n_{m_\mathrm{min}}} \ket{0}
\end{align*}
where $n_m$ is either 0 or 1, indicating whether the state with angular momentum $\hbar m$ is occupied or not, and $\ket{0}$ is the vacuum state.

We work in a Hilbert space with a fixed number of electrons $N$, so we only consider basis states where the sum of the occupation numbers equals $N$:
\[\sum_{m=m_\mathrm{min}}^{m_\mathrm{max}} n_m = N.\]

In addition since we are studying a system with a fixed total angular momentum $L$, we only consider basis states where the total angular momentum equals $L$:
\[\sum_{m=m_\mathrm{min}}^{m_\mathrm{max}} m \, n_m = L.\]

(This truncates the size of our Hilbert space even further, which is important for computational feasibility.)

Now that we have defined our many-body basis, we can construct the Hamiltonian matrix by calculating the matrix elements $\braket{\alpha|H|\beta}$, where $\ket{\alpha}$ and $\ket{\beta}$ are basis states in our chosen many-body Hilbert space.

considering the anti-commutation relations of the creation and annihilation operators:
\[\{c_m, c_n^\dagger\} = \delta_{mn}, \quad \{c_m, c_n\} = \{c_m^\dagger, c_n^\dagger\} = 0.\]
we have a complete rule book to construct the many-body Hamiltonian matrix. It is just a matter of carefully applying these rules to evaluate the matrix elements. 
This is where Claude's help can save us a lot of time and effort.

% \bibliographystyle{plain}
% \bibliography{references}

\end{document}
